\documentclass[11pt,a4paper]{jarticle}
%
\usepackage[dvipdfmx]{graphicx,color}
\usepackage{amsmath,amssymb}
\usepackage{bm}
\usepackage{graphicx}

\usepackage{caption}
\captionsetup[figure]{format=plain, labelformat=simple, labelsep=period, font=normalsize}
\usepackage{chngcntr}
\counterwithin{figure}{section}
\counterwithin{equation}{section}

\usepackage{ascmac}
\usepackage[top=30truemm,bottom=30truemm,left=30truemm,right=30truemm]{geometry}
\usepackage{indentfirst}
\usepackage{fancyhdr}
%
\pagestyle{fancy}
\lhead{}
\rhead{}
\cfoot{\thepage}
%

\title{巨大衝突ステージにおける惑星-微惑星間の力学的摩擦と微惑星間の衝突・破壊の関係}


\begin{document}
\maketitle


太陽系の地球型惑星は、最終段階で火星サイズの原始惑星同士が衝突合体を繰り返し形成されたと考えられており、この進化段階は巨大衝突ステージと呼ばれる。
巨大衝突ステージにおいて原始惑星の衝突合体により実際に地球型惑星が形成可能なことは、過去の理論研究や私の1億年分の$N$体計算の結果から確かめられた。
しかし、原始惑星同士が軌道交差を起こすほど軌道が歪んだ状態で衝突合体が起きるため、最終的に形成される地球型惑星の離心率($\sim0.1$)は、現在の太陽系の地球型惑星の離心率($\sim0.01$)を説明できない[1]。
これに関する現在の惑星形成論の理解では、地球型惑星が形成された後に残存する微惑星との力学的摩擦によって地球型惑星の離心率が下げられるという説が有力である[2]。
ところが、この力学的摩擦によって微惑星の離心率が上がるため、微惑星同士の破壊的な衝突が起きるほど相対速度は速くなり、この破壊現象によって微惑星円盤の面密度は減少していく[3]。
微惑星の面密度減少によって力学的摩擦の効率が下がるため、本当に地球型惑星の離心率を下げることが可能なのかという問題が残る。
\par
この問題を調べるためには、長期的軌道進化と破壊を扱うことができる計算が必要である。
しかし破壊によって生じる様々なサイズの微惑星は$10^{35}$個以上にもなり、$N$体計算ではとても扱うことはできない。
このような多数の粒子を取り扱うには統計力学に基づいた統計的手法が有効であるが、統計的手法では重力相互作用の取り扱いができない。
すなわち$N$体計算と統計的手法を同時に用いると、軌道進化と破壊を同時に考慮した計算を行うことができる。
\par
そこで本研究では、$N$体計算と統計的手法を組み合わせた、衝突破壊を扱うことができるハイブリッドコードを開発した。
さらに本講演では、このコードにより得られた、巨大衝突ステージにおける力学的摩擦と破壊現象の関係についても議論する。
\par
798字
\par

\begin{thebibliography}{9}
 \bibitem{1} Chambers, J.~E. \& Wetherill, G.~W. 1998, Icarus, 136, 304
 \bibitem{2} Morishima, R., Stadel, J., \& Moore, B. 2010, Icarus, 207, 517
 \bibitem{3} Kobayashi, H., \& Tanaka, H. 2010, Icarus, 206, 735
\end{thebibliography}


\end{document}